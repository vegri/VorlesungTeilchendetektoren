In $e^+e^-$-, $pp$- oder $p\bar{p}$-Kollisionen wird in den zugehörigen Detektoren der Spurverlauf
(oft) im Magnetfeld gemessen. Der Impuls der Teilchen kann dann aus der Sagitta berechnet werden:

\begin{figure}[H]
		\centering
		\includesvg[svgpath=bilder/4/]{sagitta2}
\end{figure} 

\[S= R-\sqrt{R^2-\left(\frac{L}{2} \right)^2} = R-R\cdot\text{cos}\frac{\Theta}{2}\approx R\cdot
\frac{\Theta^2}{8} \]

Mit $\Theta = e\cdot B\cdot L/p$ folgt

\[S= \frac{e\cdot\beta\cdot L^2}{8\,p}~~~~~~\text{mit}~~~p=e\cdot\beta\cdot R. \]

Wird die Teilchentrajektorie im Magnetfeld an $N$ äquidistanten Punkten gemessen, so ist der
Impulsfehler aufgrund der Ortsmessgenauigkeit $\sigma(x)$ durch die sogenannte Glückstern-Formel
gegeben:

\[ \frac{\sigma^{\text{Ort}}(p)}{p} = \frac{\sigma(x)}{0{,}3\cdot\beta\cdot L^2}\cdot
\sqrt{\frac{720}{N+4}}\cdot p \]

mit den Einheiten $\sigma(x)\left[\text{mm} \right]$, $L\left[\text{m} \right]$,
$\beta\left[\text{T} \right]$ und $p\left[\text{GeV/c} \right]$.
\\
Da die $N$ Punkte äquidistant über $L$ verteilt sind, zeigt sich

\[ \frac{\sigma^{\text{Ort}}(p)}{p} \sim \frac{1}{L^{5/2}}\cdot \frac{1}{\beta}\cdot p .\]

Das heißt, eine Vergrößerung von $L$ verbessert die Impulsauflösung besser als ein stärkeres
Magnetfeld. Die Vielfachstreuung trägt auch bei solchen Detektoren

\[\frac{\sigma^{\text{VS}}(p)}{p} = \frac{0{,}05}{\beta \cdot L}\cdot \sqrt{\frac{1{,}43\cdot
L}{x_0}} \]

bei. 
\\
Bisher wurde nur der Impuls senkrecht zu $\beta$ bestimmt. Interessant ist aber der Gesamtimpuls
$p$.  

\begin{figure}[H]
		\centering
		\includesvg[svgpath=bilder/4/]{gesamtimpuls}
\end{figure} 

Durch die Messung von $\Theta_1$ und $p_T$ (Impuls in der Transversalebene zum Mag\-netfeld) folgt

\[p=\frac{p_T}{\text{sin}\,\Theta} .\]

\begin{figure}[H]
	\centering
	\includegraphics[width=\textwidth]{collidertab.jpg}
\end{figure}