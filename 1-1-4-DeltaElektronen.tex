\FloatBarrier

Die kinetischen Energien der Elektronen ist proportional zu $\frac{1}{\Delta E^2}$.
$\Delta E$ ist dabei der Energieübertrag des Projektils auf das Hüllenelektron.

\[\frac{d\sigma}{d\Delta E} = \frac{2\pi\cdot z^2\cdot \alpha^2\cdot \hbar^2}{\beta^2\cdot m_e}\cdot
\frac{1}{\Delta E^2}
\]

Die Ausläufer dieser Verteilung gehen bis $\Delta E_\text{max}$,

\[\Delta E_\text{max} = \frac{2m_e\cdot c^2\cdot \beta^2\cdot \gamma^2}{1+ 2\,\gamma\,
\frac{m_e}{M}+\left( \frac{m_e}{M} \right)^2}\]

mit den Grenzwerten
\begin{itemize}
  \item $\gamma\rightarrow\infty$: $\Delta E_\text{max} \rightarrow \gamma Mc^2$
  \item $m_e = M$: $\Delta E_\text{max} =m_e c^2 (\gamma -1) = E- m_e c^2$ \\
  		Die komplette Energie wird auf das Hüllenelektron übertragen.
\end{itemize}

Für relativistische Teilchen ($\gamma >> 1$) kann $\Delta E_\text{max}$ sehr groß werden und
Elektronen mit Energien von einigen keV hervorbringen. In Detektoren mit hoher Granularität und Ortsauflösung (z.B.
Blasenkammern) können sie als sogenannte Delta-Elektronen nachgewiesen werden.
\\
Seltene Abstrahlung hoher Energie führt auch zu größeren Fluktuationen in $\frac{\mathrm{d}E}{\mathrm{d}x}$-Messungen
zur Teilchenidentifikation und damit zu einer schlechteren Auflösung. Auch bei Detektoren mit
geringer Granularität führt die Abstrahlung von Delta-Elektronen zu einer schlechteren
Ortsauflösung.
\\
Eine genaue Rechnung zeigt: 

\[\Delta E (\Theta) = \frac{2\,m_e}{\text{tan}^2\Theta} \]

Größere Abstrahlungswinkel führen zu kleineren Energien der Delta-Elektronen, was in einer
geringeren Reichweite resultiert.
\FloatBarrier