Die kinetischen Energien der Elektronen ist proportional zu $\frac{1}{\Delta E^2}$. $\Delta E$ ist
dabei der Energieübertrag des Projektils auf das Hüllenelektron.

\[\frac{d\sigma}{d\Delta E} = \frac{2\pi z^2 \alpha^2 \hbar^2}{\beta^2 m_e}\cdot \frac{1}{\Delta
E^2}
\]

Die Ausläufer dieser Verteilung gehen bis $\Delta E_\text{max}$,

\[\Delta E_\text{max} = \frac{2m_e c^2 \beta^2 \gamma^2}{1+ 2\gamma
\frac{m_e}{M}+\left( \frac{m_e}{M} \right)^2}
\]

mit den Grenzwerten
\begin{itemize}
  \item $\gamma\rightarrow\infty$: $\Delta E \rightarrow \gamma Mc^2$
  \item $m_e \rightarrow M$: $\Delta E \rightarrow m_e c^2 (\gamma -1) = E- m c^2$ \\
  		Die komplette Energie wird auf das Hüllenelektron übertragen.
\end{itemize}

Für relativistische Teilchen können diese Ausläufer sehr groß sein und Elektronen mit Energien von
einigen keV hervorbringen. In Detektoren mit hoher Granularität und Ortsauflösung (z.B.
Blasenkammern) können sie als sogenannte Delta-Elektronen nachgewiesen werden.
\\
Seltene Abstrahlung hoher Energie führt auch zu größeren Fluktuationen in $\frac{dE}{dx}$-Messungen
zur Teilchenidentifikation und damit zu einer schlechteren Auflösung. Auch bei Detektoren mit
geringer Granularität führt die Abstrahlung von Delta-Elektronen zu einer schlechteren
Ortsauflösung.
\\
Eine genaue Rechnung zeigt: 

\[\Delta E (\Theta) = \frac{2m_e}{\text{tan}^2\Theta} \]

Größere Abstrahlungswinkel führen zu kleineren Energien der Delta-Elektronen, was in einer
geringeren Reichweite resultiert.
