\FloatBarrier
Wie kann die Energieverteilung des Energieverlustes für ein gegebenes "`$\beta\cdot\gamma$"' beschrieben
werden? Der Energieverlust ist ein statistischer Prozess mit einer asymmetrischen
Verteilungsfunktion, da Kollisionen mit kleinem Energieübertrag wahrscheinlicher sind als solche
mit großen Energieübetrag.

\begin{figure}
	\centering
	\includegraphics[width=0.5\textwidth]{landau.jpg}
	\caption{Die Energieverteilung des Energieverlustes ist eine Landau-Verteilung.}
	\label{}
\end{figure}

Der Ausläufer bei hohen Energieüberträgen kommt von (selten auftretenden) Kollisionen mit kleinen
Stoßparametern, wobei Elektronen mit großen Ener\-gien (keV), sogenannte $\delta$-Elektronen,
freigesetzt werden. Die Asymmetrie kommt daher, dass der mittlere Energieverlust höher
ist als der wahrscheinlichste Energieverlust.
\\
Bei dünnen Absorbern wird der Energieverlust durch eine Landau-Verteilung beschrieben, bei dicken
Absorbern geht diese über in eine Gauß-Verteilung.

\FloatBarrier