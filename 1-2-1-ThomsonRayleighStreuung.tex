Diese Art der Streuung findet nur bei niederenergetischen Photonen statt. Die Thomson-Streuung
ist die Streuung von Photonen an freien Elektronen im klassischen Limit, wobei

\[ \sigma_0= \frac{8\pi}{3}\cdot r_e^2 \]

Bei der Rayleigh-Streuung streuen die Photonen am gesamten Atom, wobei alle Hüllenelektronen in
kohärenter Form beteiligt sind (kohärente Streuung).
\\
Für beide Prozesse gilt, dass kein Energieübertrag auf das Medium stattfindet. Bei hohen Energien
sind sie vernachlässigbar.
