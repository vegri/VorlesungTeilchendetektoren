Zusätzlich zur Coulombkraft $q\vec{E}$ wirkt nun auch noch die Lorentzkraft $q\vec{v}\times
\vec{B}$. Unter alleiniger Wirkung eines $B$-Feldes vollführt ein geladenes Teilchen eine
Kreisbewegung mit der Frequenz

\[\vec{\omega} = -\frac{q\vec{B}}{m}\]

mit der Elektron-Zyklotronfrequenz $\vec{\omega}$ und
$\frac{\omega}{B}=17{,}6\,\frac{\text{MHz}}{\text{Gauß}}$. Treten $\vec{E}$- und $\vec{B}$-Felder
gemeinsam auf, kann die dann schraubenförmige Bewegung in eine Rotationsbewegung mit Kreisfrequenz
$\omega$ und eine Translationsbewegung mit Geschwindigkeit $v_D$ zerlegt werden
($\vec{r}\,\perp\,\vec{v}_D$). 

\begin{figure}[H]
		\centering
		\includesvg[svgpath=bilder/3/]{rotation}
\end{figure}

\[ \vec{v}= \vec{v}_D + \vec{\omega}\times\vec{r} \]
 
 mit
 
 \[\vec{v}_D=\vec{E}\times \vec{B}/\beta^2 +\vec{v}_\shortparallel  \]
 \[\dot{\vec{v}}_\shortparallel = \frac{q}{m} \cdot\vec{E}_\shortparallel\]
 
 wobei $\vec{v}_\shortparallel$ und $\vec{E}_\shortparallel$ die Komponenten parallel zu $\vec{B}$
 sind.
 \\
 In Gasen kommet es zudem zu Stößen mit den Gasmolekülen, d.h. man muss die stochastische Kraft
 "`$m\cdot\vec{a}(t)$"' berücksichtigen. Dies führt zu der Bewegungsgleichung:
 
 \[m\cdot \dot{\vec{v}} = q\left( \vec{E}+\vec{v}\times\vec{B} \right) +m\cdot
 \vec{a}(t)~~~~~~~~~~\text{Langevin-Gleichung} \]

 Im zeitlichen Mittel gilt
 
 \[\langle \vec{a}(t) \rangle = - \dot{\vec{v}}_D  \approx \frac{\vec{v}_D}{\tau}. \]
 
 Im zeitlichen Mittel muss aus der Gleichung die Translationsbewegung mit konstanter
 Driftgeschwindigkeit folgen, d.h. die stochastische Beschleunigung $\vec{a}$ muss im Mittel gerade
 die translatorische Beschleunigung $\dot{\vec{v}}_D$ kompensieren.
