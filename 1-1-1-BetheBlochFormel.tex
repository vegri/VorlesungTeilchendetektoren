Zur klassischen Herleitung der Bethe-Bloch-Formel betrachten wir den Energieverlust $\frac{dE}{dx}$
eines schweren (d.h. $m>>m_e$), geladenen Teilchens durch Streuung an einem Hüllenelektron eines
Targetatoms.
\\
Dabei gelten folgende Annahmen:
\begin{itemize}
  \item das Hüllenelektron ist in Ruhe (Vernachlässigung der Bahnbewegung und des Rückstoßes),
  \item der Energieübertrag ist sehr viel größer als die Bindungsenergie eines Hüllenelektrons.
\end{itemize}


{\Huge ZEICHNUNG}

Der Impulsübertrag ist das Zeitintegral der durch das elektrische Feld des Projektils auf das
Target einwirkenden Kraft. Für die longitudinalen bzw. transversalen Komponenten des E-Feldes gilt

\[E_l(-x)=-E_l(x)~~~~~~~~~~~~~~~~~~~E_t(-x)=E_t(x)\]

d.h. nur der Transversalanteil ist wichtig, die longitudinalen Komponenten im Impulsübertrag heben
sich auf. Es gilt daher

\[\Delta p= \int_{-\infty}^{\infty}F\cdot dt = \int_{-\infty}^{\infty}eE_t\cdot dt =
e\int_{-\infty}^{\infty}E_t\frac{dt}{dx}\cdot dx
=e\int_{-\infty}^{\infty}E_t\frac{1}{v}\cdot dx
=\frac{e}{v}\int_{-\infty}^{\infty}E_t\cdot dx\]

Mit Gauß $\int_{-\infty}^{\infty}E_t2\pi bdx=4\pi ze$ folgt

\[\Delta p = \frac{2ze^2}{vb}\]

und somit für den Energieübertrag

\[\Delta E=\frac{\Delta p^2}{2m_e}=\frac{2z^2e^4}{m_ev^2b^2} .\]

Eine Elektronendichte von $n_e$ ergibt daher einen Energieverlust von 

\[-dE(b)=\Delta E(b)n_edv=\frac{2z^2e^4}{m_ev^2b^2}n_e2\pi b db dx\]

Nach der Integration von $b_min$ bis $b_max$ erhält man daraus:

\[-\left(\frac{dE}{dx}\right)=\frac{4\pi z^2
e^4}{m_ev^2}n_e\text{ln}\left(\frac{b_{\text{max}}}{b_{\text{min}}}\right)\]

Nun müssen nur noch $b_{\text{min}}$ bis $b_{\text{max}}$ abgeschätzt werden. $b_{\text{min}}$ wird über das
kinematische Limit abgeschätzt: Eine frontale Kollision liefert den maximalen Energieübertrag

\[\Delta E_{\text{max}}=\frac{1}{2}m_e(2v)^2\gamma^2.\]

Mit der oben abgeleiteten Beziehung

\[\Delta E(b)=\frac{2z^2e^4}{m_ev^2b^2}\overset{!}{=}\Delta E_{\text{max}}\]

ergibt sich daraus:

\[b_{\text{min}}=\frac{ze^2}{\gamma m_e v^2}.\]

Die Abschätzung von $b_{\text{max}}$ folgt aus der "`adiabatischen Invarianz"': Die Targetelektronen
sind in Atomen gebunden und "`umkreisen"' die Atomkerne mit einer mittleren Orbitalfrequenz
$\overline{\nu}$. Damit ein Energieübertrag stattfindet, muss die Zeitdauer der Störung, $\Delta t$,
kürzer sein als die Periodendauer $\tau$:

\[\Delta t=\frac{b}{\gamma v} \le \tau =\frac{1}{\overline{\nu}}\]

Daraus folgt

\[b_{\text{max}}=\frac{\gamma v}{\overline{\nu}}.\]

Jetzt führen wir noch eine Größe for die Elektronen-Dichte des Target-Materials ein:

\[n_e=N_A\cdot \rho\cdot \frac{Z}{A}\]

mit der Avogadrozahl $N_A$, der Targetdichte $\rho$, der Ordnungszahl $Z$ und der Massenzahl $A$.
Einsetzen der Grenzen für den Stoßparameter in die Formel und Substitution von $n_e$ führt zu

\[-\left(\frac{dE}{dx}\right)_{\text{coll}} = \frac{4\pi z^2e^4}{m_ev^2}N_A\cdot \rho
\frac{Z}{A}\cdot\text{ln}\left(\frac{\gamma^2 m_e v^3}{2e^2\overline{\nu}}\right), \]

was der klassischen Formel von Bohr entspricht. Diese beschreibt den Energieverlust für schwere
Teilchen (Protonen, $\alpha$-Teilchen,\ldots) durch Anregung und Ionisation. Für leichte Teilchen
müssen Quanteneffekte berücksichtigt werden.
\\
Die quantenmechanische Rechnung führt zur Bethe-Bloch(-Sternheimer)-Formel:

\[-\left(\frac{dE}{dx}\right)_{\text{coll}} = 2\pi N_A r_e^2 m_e c^2 \rho \frac{Z}{A}
\frac{z^2}{\beta^2}\left[ \text{ln} \left( \frac{2m_e c^2 \gamma^2 \beta^2 W_{\text{max}}}{I^2}
\right) -2\beta^2 -\delta -2\frac{c}{z} \right]\]

mit 

\[\beta =
\frac{v}{c},~~~~~~~~~~~~\gamma=\frac{1}{\sqrt{1-\beta^2}}~~~~~~~~~~~~
r_e=\frac{1}{4\pi\epsilon_0}\cdot\frac{e^2}{m_e c^2}\]

sowie
\begin{description}
\item[$z$]Ladung des einfallenden Teilchens
\item[$Z, A$] Ordnungs- und Massenzahl des Targets
\item[$\rho$] Targetdichte
\item[$N_A$] Avogradozahl
\item[$I$] mittleres Ionisationspotential (Materialkonstante des Targets)
\item[$W_{\text{max}}$] max. Energieübertrag in einer Einzelkollision
\item[$\delta$] Dichtekorrektur (Polarisationseffekt, $\delta \approx 2\text{ln}(\gamma)+K$)
\item[$c$] Schalenkorrektur (wichtig für kleine Projektilgeschwindigkeiten)
\end{description}


Anmerkungen zur Bethe-Bloch-Formel:
\begin{itemize}
  \item sie beschreibt den Energieverlust sehr gut im Bereich $0,1 < \gamma\beta < 100$;
  \item es gibt drei Bereiche:
  			\begin{itemize}
  			  \item bei niedrigen Energien gibt es einen Abfall bis zu einem Minimum (bei $\gamma\beta$
  			  ca. 3-3,5), Teilchen an diesem Punkt sind minimal ionisierende Teilchen;
  			  \item danach beginnt ein logarithmischer Anstieg mit zunehmender Teilchenenergie, der
  			  sogenannte "`relativistische Anstieg"';
  			  \item bei hohen Energien wird das Fermi-Plateau erreicht: durch Polarisationseffekte erreicht
  			  der Energieverlust  einen Sättigungswert (Dichtekorrektur);
  			  \end{itemize}
  \item beim Energieverlust handelt es sich um einen statistischen Vorgang.
\end{itemize}

Die Bethe-Bloch-Formel beschreibt den mittleren Energieverlust durch Ionisation und Anregung. Sie
gilt für alle geladenen Teilchen außer Elektronen und Positronen. Für diese muss die Gleichheit der
Massen und die Ununterscheidbarkeit der Stoßpartner berücksichtigt werden. Unsere Ableitung
unterscheidet sich von der Bethe-Bloch-Formel numerisch durch einen Faktor 2, was durch eine
mangelhafte Berücksichtigung von Fernstößen zustande kommt. Für die quantenmechanische Beschreibung
des Energieverlustes gibt es verschiedene Varianten der $\frac{dE}{dx}$-Formel, was an der
unterschiedlichen Parametrisierung der Fernstöße, d.h. jenes Energieverlustes, bei dem die Bindung
der Elektronen in den Atomhüllen nicht vernachlässigbar ist.
\\
Meist wird der Energieverlust pro Wegstrecke $\frac{1}{\rho}\frac{dE}{dx}$ angegeben, wobei $\rho$
die Dichte in $\frac{\text{g}}{\text{cm}^3}$ ist. $\frac{1}{\rho}\frac{dE}{dx}$ ist für ein MIP nur
schwach vom Absorbermaterial abhängig und beträgt ca. 

\[2 \text{MeV}\frac{\text{cm}^2}{\text{g}}.\]