\FloatBarrier
Übergangsstrahlung tritt auf, wenn ein geladenes Teilchen die Grenzfläche zweier
Materialien mit unterschiedlicher Dielektrizitätskonstante $\epsilon$ durchquert.
\\
Bei Material mit niedrigem $\epsilon$ ist die Polarisation des Materials klein, das elektrische Feld
der bewegten Ladung hat eine große räumliche Ausdehnung. Bei Material mit hohem $\epsilon$ hingegen
ist die Polarisation der Materials groß, das elektrische Feld der bewegten Ladung hat somit eine
geringe räumliche Ausdehnung. Die plötzliche Umverteilung der Ladungen an der Grenzfläche und die
daraus resultierende schnelle Änderung des elektrischen Feldes ist die Ursache der
Übergangsstrahlung.
\\
Übergangsstrahlung wird hauptsächlich als Röntgenstrahlung emittiert. Die Emissionsrichtung liegt in
der Bewegungsrichtung des Projektils innerhalb eines Konus mit dem Öffnungswinkel

\[\text{cos}\,\Theta_t \approx \frac{1}{\gamma}~~~~~\text{mit}~~~\gamma=\frac{1}{\sqrt{1-\beta^2}} 
\]

Tritt ein Teilchen mit der Ladung $z\cdot e$ vom Vakuum in ein Medium mit der Plasmafrequenz
$\omega_p$ über, so liegt die als Übergangsstrahlung emittierte Energie bei

\[E_t = \frac{1}{3}\, \alpha\cdot z^2\cdot \gamma\cdot \hbar\cdot\omega_p. \]

Mittels Energiemessung der Übergangsstrahlung kann man $\gamma$ und somit die
Projektilgeschwindigkeit messen. Für eine typische Photonenergie von $\gamma\cdot
\hbar\cdot\omega_p/4$ ist die mittlere Anzahl von emittierten Photonen pro Grenzfläche
(=~Quantenausbeute) ungefähr

\[ \langle N \rangle = \frac{2}{3}\,\alpha\cdot z^2~~~~~\text{mit}~~~~\alpha\sim \frac{1}{137}.  \]

\FloatBarrier