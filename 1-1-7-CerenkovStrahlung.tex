\v{C}erenkov-Strahlung wird emittiert, wenn die Geschwindigkeit $v$ eines Teilchens größer ist als
die Lichtgeschwindigkeit in dem vom Teilchen durchquerten Material:

\begin{figure}[H]
	\centering
	\includegraphics[width=0.5\textwidth]{dummy.jpg}
	\caption{Veranschauulichung der Entstehung von \v{C}erenkov-Strahlung. Links $v<c/n$, rechts $v>cn$
	(elektromagnetische Schockwelle) mit $c=$ Vakuumlichtgeschwindigkeit und $n=$ Brechungsindex. }
\end{figure}

Bewegt sich ein geladenes Teilchen durch ein dielektrisches (nichtleitendes) Medium, kommt es zu
einer kurzzeitigen Polarisierung der Atome längs der Flugbahn durch dessen Ladung. Dabei werden
elektromagnetische Wellen erzeugt. Im Normalfall inteferieren die Wellen destruktiv mit den Wellen
von benachbarten Atomen. Ist das Teilchen jedoch schneller als das Licht in diesem Medium, können
sich die Wellen nicht mehr gegenseitig auslöschen. Es entsteht eine gemeinsame kegelförmige
Wellenfront, die \v{C}erenkov-Strahlung.
\\
Die kohärente Wellenfront hat konische Form und wird unter einem Winkel von 

\[\text{cos}\,\Theta_c=\frac{1}{\beta\cdot n}~~~~~\text{mit}~~\beta=\frac{v}{c}\]

abgestrahlt. \v{C}erenkov-Strahlung ist nur für ca 1\% des Energieverlustes eines Teilches
verantwortlich. Die Zahl der pro Weglänge $L$ abgestrahlten Photonen beträgt

\[\frac{dN_{ph}}{dL}= 2\pi\cdot\alpha\cdot z^2 \int_{\lambda_2}^{\lambda_1}
\frac{\overbrace{1-1/(\beta^2n^2)}^{\text{sin}^2\,\Theta}}{\lambda^2}\,\mathrm{d}\lambda\]

mit der Ladung $z$ des Teilchens und der Wellenlänge $\lambda$ der Strahlung. 
\\
Unter der Annahme, dass $n(\lambda)$ im entsprechenden Wellenlängenbereich konstant ist, gilt

\[\frac{dN_{ph}}{dL}= 2\pi\cdot\alpha\cdot z^2 \, \frac{1-1/\beta^2n^2}{\lambda}\]

Die Zahl der abgestrahlten Photonen nimmt mit kleinerer Wellenlänge wie 1/$\lambda$ zu.