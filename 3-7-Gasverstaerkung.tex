Bisher wurden nur geringe $\vec{E}$-Felder betrachtet, womit der Energiegewinn der Elektronen klein
ist ($qE\lambda_e$). Sehr starke Felder ermöglichen dagegen einen Energiegewinn, der eine
Sekundärionisation möglich macht. Daraus folgt eine Zunahme der Ladungsträger.

\begin{figure}[H]
	\centering
	\includegraphics[width=0.5\textwidth]{dummy.jpg}
\end{figure}

Diese lawinenartige Zunahme führt zu einer erheblichen Ladungsmenge, die einfach gemessen werden
kann. Man nennt diesen Prozess Gasverstärkung.
\\
Die Anzahl der Elektron-Ion-Paare, die ein Elektron pro cm Wegstrecke bildet, wird durch den ersten
Townsend-Koeffizienten $\alpha$ bezeichnet. Aus dem Stoßquerschnitt $\sigma_i$ kann $\alpha$ durch 

\[\alpha = \sigma_i\cdot N~~~~~~\text{mit}~~~~~ N=\frac{N_0}{V_{\text{mol}}}\approx
2{,}69\cdot10^{19}\,\frac{\text{Atome}}{\text{cm}^3} \]

berechnet werden. Eine Elektronenlawine hinterlässt eine charakteristische Verteilung der positiven
und negativen Ladungsträgern in Tropfenform. Die höhere Beweglichkeit der Elektronen führt zu der
"`negativen"' Spitze (siehe Abb. \ref{bla}). Die eigentliche Tropfenform entsteht durch die
unbeweglichen Ionen.

\begin{figure}[H]
	\centering
	\includegraphics[width=0.5\textwidth]{dummy.jpg}
\end{figure}

Beginnend mit $n_o$ Primärelektronen am Ort $x=0$ werden durch Stoßionisation entlang von $dx$

\[ dn = \alpha\cdot n \mathrm{d}x  \]

zusätzliche Elektronen erzeugt, sodass

\[n(x)=n_0\cdot e^{\alpha\cdot x}   \]

gilt, falls $\alpha$ unabhängig von $x$ ist. Da $\alpha$ von der Feldstärke abhängt, die sich
entlang von $x$ ändern kann, gilt

\[n(x)=n_0\cdot \text{exp}\left(\int\alpha(x)\mathrm{d}x \right)=:n_0\cdot A  .\]

Dabei bezeichnet $A$ die Gastverstärkung. $A$ hängt von der freien Weglänge $\lambda$ ab:

\[\lambda =\frac{1}{\alpha}=\frac{1}{N\cdot\sigma_i}  \]

Wichtig ist hierbei: Die Anzahl der Elektronen am Ort $x$ ist proportional zu $n_0$
(Proportionalitätsbereich der Gasverstärkung).
\\
Das Ende des exponentiellen Anstiegs der Gasverstärkung wird durch UV-Photonen eingeleitet, die
durch den Photoeffekt im Gas oder aus der Kathode weitere Elektronen auslösen.

\[n_0 \text{ primäre Elektronen} \longrightarrow A\cdot n_0 \text{ Elektronen} + (A\cdot n_0)\cdot
\gamma \text{ Photoelektronen}\]
\[(A\cdot n_0)\cdot\gamma \text{ Photoelektronen} \longrightarrow A(A\cdot n_0)\cdot \gamma
\text{ Elektronen} + A(A\cdot n_0)\cdot \gamma^2 \text{ Photoelektronen}\]
\ldots

Hierbei bezeichnet $\gamma$ den zweiten Townsend-Koeffizenten.

Die Gasverstärkung einschließlich der Energieausbreitung durch Photonen ist


\begin{align*}
A_\gamma :&= n_0\cdot A + n_0\cdot A^2\cdot\gamma + n_0\cdot A^3\cdot\gamma^2 + \ldots\\
&=n_0\cdot A \sum_{\nu\geq0}(A\cdot\gamma)^\nu  \\
&= \frac{n_0\cdot A}{1- A\cdot\gamma}
\end{align*}

wenn $A\cdot\gamma<1$. Somit ist $A_\gamma\sim n_0$.
\\
Für $A_\gamma\rightarrow 1$ wird $A\gamma$ unabhängig von $n_0$. In diesem sogenannten
Auslösebereich wird $\alpha\cdot x\approx 20$ oder $A\approx10^8$ erreicht.

\begin{figure}[H]
	\centering
	\includegraphics[width=0.65\textwidth]{Fig-03-16.jpg}
	\caption{Abhängig von der $|\vec{E|}$-Feldstärke können verschiedene Bereiche der Gasverstärkung
	identifiziert werden.}
\end{figure}

\subsubsection*{Berechnung erster Townsend-Koeffizient}

Eine einfache, für kleine $\alpha$ gültige Approximation des ersten Townsend-Koeffizienten $\alpha$
stammt von S. Korff:

\[\alpha \approx p\cdot A\cdot e^{-\beta p/E}  \]

mit dem Druck $p$ und der Feldstärke $E$. In diesem Bereich ist $\alpha$ zudem linear abhängig von
der kinetischen Elektronenergie $E_\text{kin}$

\[\alpha \approx k\cdot N \cdot E_\text{kin} ~~~~~~ \text{mit}~~~~~N\approx 2{,}69\cdot
10^{19}\,\frac{\text{Atome}}{\text{cm}^3}.\]

Die Parameter $A, \beta$ und $k$ sind z.B.:

\begin{figure}[H]
	\centering
	\includegraphics[width=0.5\textwidth]{Fig-03-17.jpg}
\end{figure}

\subsubsection*{Zusammenfassung}

Ionisation und Anregung erzeugen freie Ladungsträger im Gas. Diese Ladungsträger können sich durch
Diffusion oder unter der Wirkung elektrischer Felder im Gas gemäß ihrer Beweglichkeit $\mu$
bewegen. Elektronegative Atome lagern dabei freie Elektronen an. Im elektrischen Feld stellt sich
eine konstante Driftgeschwindigkeit $v_D\sim E\lambda_e/T$ ein, ihr Wert hängt vom
Elektron-Gasatom-Stoßwirkungsquerschnitt ab (Ramsauer-Minimum). Im elektrischen und magnetischen
Feld ergibt sich eine schraubenförmige Bewegung. Diese Felder beeinflussen auch die Diffusion.
\\
Sehr starke elektrische Felder führen zur Gasverstärkung mit leicht messbaren Ladungsmengen.
