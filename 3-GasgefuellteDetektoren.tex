Die relevanten Prozesse für den Energieverlust in gasgefüllten Detektoren sind Anregung und
Ionisation.
\\
Die Anregung eines Atoms $A$ durch ein geladenes Teilchen $x$ der Form

\[x+A \longrightarrow x+ A^*\]

erfordert eine genaue Energiemenge beim Übertrag. Typische Wirkungsquerschnitte liegen
im Bereich $\sigma_{\text{Anregung}}=10^{-17}\text{cm}^2$. Die Anregungsenergie kann über folgende Prozesse
abgegeben werden:

\begin{itemize}
  \item Strahlung:\\ $A^*\longrightarrow A+h\nu$
  \item Kollision:\\ z.B. $\text{Ne}^*+\text{Ar}\longrightarrow \text{Ne}+\text{Ar}^++e^-$
  \item Ion-Molekülbildung (Edelgase):\\ $\text{He}^*+\text{He} \longrightarrow\text{He}^+_2+e^-$
\end{itemize}

Die Ionisation eines Atoms $A$ der Form

\[x+A \longrightarrow x+ A^+ +e^- \]

erfordert dagegen keinen genauen Energieübertrag. Typische Wirkungsquerschnitte liegen hierbei im
Bereich von $\sigma_{\text{Ion}}=10^{-16}\text{cm}^2$, was größer ist als
$\sigma_{\text{Anregung}}$. Da aber ein großer Energieübertrag erforderlich ist, um ein Atom zu
ionisieren, während kleine Energieüberträge häufiger sind, dominiert die Anregung über die Ionisation.
\\
Es gibt zwei Arten der Ionisation:

\begin{itemize}
  \item primäre Ionisation:\\ $x+A \longrightarrow x+A^++e^-$
  \item sekundäre Ionisation:\\ $x+A \longrightarrow x+A^++\delta_{e^-}$ \\ $\delta_{e^-}+A \longrightarrow
  \delta_{e^-}+A^++e^-$
\end{itemize}
